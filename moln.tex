\documentclass{article}
\usepackage{parskip}
\usepackage[dvipsnames]{xcolor}

\newcommand{\cmd}[1]{\textcolor{Red}{#1}}
\newcommand{\help}[1]{#1\par}
\newcommand{\cloud}[1]{\textcolor{Green}{#1}\\}

\begin{document}
\tt
    {\Large Moln Cross Reference CLI Commands AWS/Azure/Gcloud}

    Moln is a bash shell script to simplify the initial work with multiple cloud providers.

    \begin{itemize}
    \item{Use it to quickly install the cloud providers cli}
    \item{Use it to perform commands on several cloud providers at once.}
    \item{Use it to learn how to work with different cloud provicer cli:s.}
    \end{itemize}

    Perform specific cloud configuration using the web interface (for example vm templates)
    then use moln to start a vm using the prepared vm template.

    Examples:

    \textcolor{Blue}{moln aws install} \linebreak
    \textcolor{Blue}{moln azure whoami} \linebreak
    \textcolor{Blue}{moln gcloud list-users} \linebreak
    \textcolor{Blue}{moln aws list-subnets} \linebreak
    \textcolor{Blue}{moln all list-vms} \linebreak
    \textcolor{Blue}{moln aws list-vm-templates} \linebreak
    \textcolor{Blue}{moln aws create-vm-from-templates} \linebreak
    \textcolor{Blue}{moln aws upsert-dns-record www.fooobar.com A 10.11.12.13} \linebreak
    \textcolor{Blue}{moln aws assume-role allaccess MYSESSION3} \linebreak
    \textcolor{Blue}{moln --list-help} \linebreak
    \par{\large BASIC}\par
\cmd{whoami} \help{Display active account/identity.}
\cloud{aws sts get-caller-identity}
\cloud{az account list}
\cloud{gcloud config list}

{\large BUCKET}\par
\cmd{list-buckets} \help{List storage buckets, ie aws s3/gcloud storage/azure blobs.}
\cloud{aws s3api list-buckets --output json}

{\large COST}\par
\cmd{list-costs} \help{Print a summary of monthly costs for the last 12 months.}

{\large DNS}\par
\cmd{list-hosted-zones} \help{List zones the dns is configured to serve.}
\cloud{aws route53 --region us-east-1 list-hosted-zones}
\cmd{list-dns-records} \help{List dns records in a zone.}
\cloud{aws route53 list-resource-record-sets --hosted-zone-id=\${ID}}
\cmd{upsert-dns-record NAME TYPE DEST} \help{Insert or update a dns record.}
\cloud{aws route53 change-resource-record-sets --hosted-zone-id \${ID} --change-batch file://\${FILE}}
\cmd{remove-dns-record NAME} \help{Remove a dns record.}
\cloud{aws route53 change-resource-record-sets --hosted-zone-id \${ID} --change-batch file://\${FILE}}
\cmd{list-domains} \help{List dns domains.}
\cloud{aws route53domains --region us-east-1 list-domains}

{\large GROUP}\par
\cmd{list-groups} \help{List iam groups.}
\cloud{aws iam list-groups}
\cloud{az ad group list}
\cmd{list-groups-for-user} \help{List groups to which a user belongs.}

{\large CLI}\par
\cmd{install} \help{Install the specified cloud cli.}

{\large IP}\par
\cmd{list-ips} \help{List allocated external ip numbers.}
\cloud{aws ec2 describe-addresses}

{\large ORG}\par

{\large POLICY}\par
\cmd{list-policies} \help{List iam policies.}
\cloud{aws iam list-policies}

{\large ROLE}\par
\cmd{assume-role ROLE SESSION} \help{Start a subshell with the rights of the assumed role.}
\cloud{aws sts assume-role --role-arn "arn:aws:iam::\${AWS\_ACCOUNT}:role/\$ROLE" --role-session-name "\$SESSION"}
\cmd{list-roles} \help{List iam roles.}
\cloud{aws iam list-roles}

{\large SUBNET}\par
\cmd{list-subnets} \help{List all subnets.}
\cloud{aws ec2 describe-subnets}

{\large USER}\par
\cmd{list-users} \help{List users in cloud account.}
\cloud{aws iam list-users}
\cloud{az ad user list}
\cloud{gcloud iam service-accounts --format=json list}

{\large VM}\par
\cmd{create-vm-from-template NAME TEMPLATE\_NAME} \help{Create a virtual machine based on an existing vm template.}
\cloud{aws ec2 run-instances --launch-template LaunchTemplateName=\${TEMPLATE\_NAME} --tag-specifications "ResourceType=instance,Tags=[{Key=Name,Value=\${NAME}}]"}
\cloud{az foo bar \${TEMPLATE\_NAME} \${NAME}}
\cmd{destroy-vm NAMEID NAMEID} \help{Destroy a virtual machine.}

{\large VPC}\par
\cmd{list-vpcs} \help{List virtual private clouds/networks, aka vpc:s and vnets.}
\cloud{aws ec2 describe-vpcs}
\cloud{az network vnet list}

{\large WEBAPI}\par
\cmd{list-webapi-domains} \help{List domain names mapped to web apis (REST/HTTPs) routers.}
\cloud{aws apigatewayv2 get-domain-names}

\end{document}